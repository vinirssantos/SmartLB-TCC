% resumo em inglês
\begin{resumo}[Abstract]
 \begin{otherlanguage*}{english}
This paper presents a load balancer proposal to reduce the execution time of  parallel applications when they run on multiprocessors environments. The load balancer algorithm collects system and application information in real time and uses them to make load balancing decisions dynamically, aiming to reduce the rate of migration and to decrease the execution time. \charm was used for implementation due to its compatibility with the desired environments. The results show that our load balancer improves performance by reducing task migration from $3$ up to $137$ times and reductions from  $3.5$\% to $19.8$\%  on total execution time.
\begin{comment}

\textbf{}
This paper deals with a topic that is much discussed at present, parallel and heterogeneous computing. It aims at comparing the performance and energy efficiency of components that fall into this category. For this, an Intel Xeon Phi coprocessor and an NVIDIA GPGPU will be compared. Before comparing them, a study of the NVIDIA GPU architectures and the Intel Xeon Phi coprocessor (Intel MIC) was needed, as well as some APIs and platforms that enable the implementation of these components. The comparison was performed using a benchmark, which was also studied. As a result, it was realized how efficient both components may be, in the case of Xeon Phi, that in the benchmark MaxFlops reached 5.06 TFlops for single precision values with an energy efficiency of 17,27 GFlops/W, and in NVIDIA GPGPU, which in the GEMM benchmark, reached 3.03 TFlops for single precision values and an energy efficiency of 13.18 GFlops/W. 

     \textbf {Keywords}: Parallel Computing, Heterogeneous Computing, High Performance Computing, CUDA, OpenCL, OpenMP, SHOC Benchmark Suite.
\end{comment}
 \end{otherlanguage*}
\end{resumo}