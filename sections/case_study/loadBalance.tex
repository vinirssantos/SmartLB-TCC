\section{Balanceamento de Carga}

O balanceamento de carga é uma técnica de distribuição de carga computacional e de comunicação uniformemente em todos os processadores de uma máquina paralela, para que nenhum processador seja sobrecarregado. As estratégias de balanceamento de carga podem ser divididas em duas categorias. Para aplicativos onde novas tarefas são criadas e programado durante a execução e aqueles para aplicações iterativas com padrões de carga persistentes~\cite{zheng2010}.

As estratégias de balanceamento de carga podem ser usadas durante a execução do aplicativo para melhorar a distribuição das tarefas. Essas estratégias tentam encontrar uma nova distribuição de tarefas que maximize o uso do núcleo, levando em consideração os tempos de execução das tarefas. Ainda assim, esta distribuição de trabalho pode não proporcionar um ótimo desempenho devido à comunicação de despesas gerais a partir do projeto multi-core de sistemas atuais ~\cite{pilla2014topology}.

De acordo com \cite{hendrickson2000dynamic}, muitas aplicações paralelas estão se movendo em direção a clusters de computadores de memória compartilhada distribuída e sistemas de computação heterogêneos. O balanceamento de carga nesses sistemas está se tornando uma área de pesquisa ativa. Uma abordagem para o balanceamento de carga nesse tipo de sistema é simplesmente mudar a atribuição de trabalho para processadores. O espaço de endereço global das máquinas pode então localizar dados quando necessário pelo aplicativo, ignorando a migração de dados complicada. No entanto, uma vez que as referências de memória aplicativo são caras, é vantajoso mover realmente os dados atribuídos para a própria memória do processador. Por razões de desempenho, então, o problema dinâmico de balanceamento de carga em sistemas de memória compartilhada compartilhada parece quase idêntico ao dos computadores MIMD. 