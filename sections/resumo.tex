\setlength{\absparsep}{18pt} % ajusta o espaçamento dos parágrafos do resumo
\begin{resumo}
	% tema
Este artigo apresenta uma proposta de balanceamento de carga para a redução do tempo de execução de aplicações paralelas executadas em ambientes multiprocessados.
% obj
O algoritmo do balanceador coleta informações do sistema e da aplicação em tempo real e as utiliza na tomada de decisões de balanceamento de carga dinamicamente, visando reduzir o número de migrações de tarefas enquanto reduzindo o tempo total de execução.  
% met
Para implementação foi utilizado o modelo de programação paralela \charm. 
% res
Os resultados preliminares apresentaram reduções de $3$ à $137$ vezes na quantidade de migrações de tarefas e reduções de $3,5$\% à $19,8$\% no tempo total de execução.	
\begin{comment}
Este trabalho trata de um tema bastante discutido atualmente, a computação paralela e heterogênea. Visa realizar a comparação de desempenho e eficiência energética de componentes que se enquadram nesta categoria. Para isso, serão comparados um coprocessador Intel Xeon Phi e uma GPGPU NVIDIA. Antes de compará-los, foi necessário um estudo das arquiteturas das GPUs da NVIDIA e do coprocessador Intel Xeon Phi (Intel MIC), bem como de algumas APIs e plataformas que possibilitam a implementação nestes componentes. A comparação foi realizada utilizando um \textit{benchmark}, que igualmente foi estudado. Como resultado, percebeu-se o quão eficientes ambos os componentes podem ser, no caso do Xeon Phi, que no \textit{benchmark} \textit{MaxFlops}, alcançou 5,06 TFlops para valores de precisão simples com uma eficiência energética de 17,27 GFlops/W e, na GPGPU NVIDIA, que no \textit{benchmark} GEMM, atingiu 3,03 TFlops para valores de precisão simples e uma eficiência energética de 13,18 GFlops/W. 


- O resumo deve responder 4 perguntas:

	a) qual o tema do trabalho; 

	b) qual o objetivo; 

	c) qual a metodologia empregada; e 

	d) qual o resultado/conclusão do trabalho.




\textbf{Palavras-chave}: Computação Paralela, Computação Heterogênea, Computação de Alto Desempenho, CUDA, OpenCL, OpenMP, SHOC \textit{Benchmark Suite}.
\end{comment}
\end{resumo}