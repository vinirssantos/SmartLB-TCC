\subsection{Storages}

Os Storages são dispositivos projetados especificamente para o armazenamento de dados. Sua utilização é empregada em diversas aplicações nos ambientes de TI, como servidores de arquivos, backup, área de compartilhamento, ou seja, tudo que esteja relacionado a administração e processamento de dados armazenados. Segundo o autor \citeonline{controle_net}, os Storages são divididos em quatro categorias, conforme suas características:
\begin{itemize}
\item \textbf{DAS (Direct Attached Storage)}

São dispositivos que precisam ser obrigatoriamente conectados diretamente ao servidor ou estação de trabalho, através de conexões como USB, eSATA, fire-wire, Thunderrbolt ou MiniSAS. Seu funcionamento é basicamente como um disco complementar de armazenamento, ou seja, uma extensão.

Destaca-se pelo o fato de que este aparelho oferece grande versatilidade para armazenamento de grandes volumes, com alta carga de trabalho que precisam de alta capacidade e alta perforance. Os storages DAS utilizam conexões MiniSAS, disponíveis nas velocidade de 6 Gbps ou 12 Gbps para alta performance, podendo realizar backup e compartilhamento de dados entre os computadores conectados.

\item \textbf{NAS (Network Attched Storage)}

São dispositivos de armazenamento de dados, ligados diretamente a Rede etherner, rodando um sistema operacional completo, funcionando como servidor de arquivos.

Realiza o armazenamento de dados via bloco, permitindo uma solução de armazenamento mais eficaz, versátil e simplificado.

\item \textbf{SAN (Storage Area Network)}
Consiste em uma estrutura de armazenamento de dados caracterizada por uma rede dedicada de armazenamento composta por servidores e storages. Os storages são ligados aos servidores através de switch ou conexão direta, usando conexão de Fibra óptica, Ethernet 1 Gbps ou Ethernet 10 Gbps. Esta estrutura não permite acesso direto por parte do usuário. É necessário acessar o servidor, que este direciona o armazenamento para o storage.

Outra grande caraterística deste sistema é que ele facilita o compartilhamento entre os servidores e simplifica as operações de TI, pois consolida o storage em um local central, garantindo uma das melhores soluções de armazenamento, conforme descrito no site \citeonline{5ti}.

\item \textbf{FAS (Flash all Storage)}

Estes são Storages desenvolvidos para trabalhar fazendo o uso somente de dispositivos SSD, sem a necessidade de utilizar HDD convencionais com discos rotacionando e partes mecânicas. São caracterizados por armazenamento total em memória flahs e ssd.

\end{itemize}

Conforme fala o autor \citeonline{controle_net}, todos esses tipos de Storages possuem tecnologia RAID, de forma que se obtenha tolerância a falha de discos, reduzindo o risco de perda de dados, garantindo que a produtividade do ambiente não seja afetada. Mesmo que o equipamento apresente falha de disco, o Storage não para de trabalhar, sendo possível realizar a substituição do HD ou SSD com defeito com o Storage em funcionamento.

Outro fator importante são os discos a serem utilizados no Storage, pois existe uma grande variedade de tipos e modelos de HDs e SSDs diferentes. Se não houver um dimensionamento correto entre disco e Storage, pode acabar resultando em prejuízos e perda de dados. O tipo de HD ou SSD a ser utilizado no storage deve sempre ser dimensionado de acordo com a necessidade de cada ambiente.

Os Storages também oferecem alta capacidade de armazenamento, através da utilização de unidades de expansão, caso a demanda aumente, basta adicionar mais HDs ou SSDs. Além do mais, existem Storages que podem ser configurados em clusters, garantindo que mesmo que houver falha de um Storage, o tabalho no ambiente não pare.


